\documentclass[]{article}
\usepackage{lmodern}
\usepackage{amssymb,amsmath}
\usepackage{ifxetex,ifluatex}
\usepackage{fixltx2e} % provides \textsubscript
\ifnum 0\ifxetex 1\fi\ifluatex 1\fi=0 % if pdftex
  \usepackage[T1]{fontenc}
  \usepackage[utf8]{inputenc}
\else % if luatex or xelatex
  \ifxetex
    \usepackage{mathspec}
  \else
    \usepackage{fontspec}
  \fi
  \defaultfontfeatures{Ligatures=TeX,Scale=MatchLowercase}
\fi
% use upquote if available, for straight quotes in verbatim environments
\IfFileExists{upquote.sty}{\usepackage{upquote}}{}
% use microtype if available
\IfFileExists{microtype.sty}{%
\usepackage{microtype}
\UseMicrotypeSet[protrusion]{basicmath} % disable protrusion for tt fonts
}{}
\usepackage[margin=1in]{geometry}
\usepackage{hyperref}
\hypersetup{unicode=true,
            pdftitle={615 Pre\_EDA},
            pdfborder={0 0 0},
            breaklinks=true}
\urlstyle{same}  % don't use monospace font for urls
\usepackage{color}
\usepackage{fancyvrb}
\newcommand{\VerbBar}{|}
\newcommand{\VERB}{\Verb[commandchars=\\\{\}]}
\DefineVerbatimEnvironment{Highlighting}{Verbatim}{commandchars=\\\{\}}
% Add ',fontsize=\small' for more characters per line
\usepackage{framed}
\definecolor{shadecolor}{RGB}{248,248,248}
\newenvironment{Shaded}{\begin{snugshade}}{\end{snugshade}}
\newcommand{\KeywordTok}[1]{\textcolor[rgb]{0.13,0.29,0.53}{\textbf{#1}}}
\newcommand{\DataTypeTok}[1]{\textcolor[rgb]{0.13,0.29,0.53}{#1}}
\newcommand{\DecValTok}[1]{\textcolor[rgb]{0.00,0.00,0.81}{#1}}
\newcommand{\BaseNTok}[1]{\textcolor[rgb]{0.00,0.00,0.81}{#1}}
\newcommand{\FloatTok}[1]{\textcolor[rgb]{0.00,0.00,0.81}{#1}}
\newcommand{\ConstantTok}[1]{\textcolor[rgb]{0.00,0.00,0.00}{#1}}
\newcommand{\CharTok}[1]{\textcolor[rgb]{0.31,0.60,0.02}{#1}}
\newcommand{\SpecialCharTok}[1]{\textcolor[rgb]{0.00,0.00,0.00}{#1}}
\newcommand{\StringTok}[1]{\textcolor[rgb]{0.31,0.60,0.02}{#1}}
\newcommand{\VerbatimStringTok}[1]{\textcolor[rgb]{0.31,0.60,0.02}{#1}}
\newcommand{\SpecialStringTok}[1]{\textcolor[rgb]{0.31,0.60,0.02}{#1}}
\newcommand{\ImportTok}[1]{#1}
\newcommand{\CommentTok}[1]{\textcolor[rgb]{0.56,0.35,0.01}{\textit{#1}}}
\newcommand{\DocumentationTok}[1]{\textcolor[rgb]{0.56,0.35,0.01}{\textbf{\textit{#1}}}}
\newcommand{\AnnotationTok}[1]{\textcolor[rgb]{0.56,0.35,0.01}{\textbf{\textit{#1}}}}
\newcommand{\CommentVarTok}[1]{\textcolor[rgb]{0.56,0.35,0.01}{\textbf{\textit{#1}}}}
\newcommand{\OtherTok}[1]{\textcolor[rgb]{0.56,0.35,0.01}{#1}}
\newcommand{\FunctionTok}[1]{\textcolor[rgb]{0.00,0.00,0.00}{#1}}
\newcommand{\VariableTok}[1]{\textcolor[rgb]{0.00,0.00,0.00}{#1}}
\newcommand{\ControlFlowTok}[1]{\textcolor[rgb]{0.13,0.29,0.53}{\textbf{#1}}}
\newcommand{\OperatorTok}[1]{\textcolor[rgb]{0.81,0.36,0.00}{\textbf{#1}}}
\newcommand{\BuiltInTok}[1]{#1}
\newcommand{\ExtensionTok}[1]{#1}
\newcommand{\PreprocessorTok}[1]{\textcolor[rgb]{0.56,0.35,0.01}{\textit{#1}}}
\newcommand{\AttributeTok}[1]{\textcolor[rgb]{0.77,0.63,0.00}{#1}}
\newcommand{\RegionMarkerTok}[1]{#1}
\newcommand{\InformationTok}[1]{\textcolor[rgb]{0.56,0.35,0.01}{\textbf{\textit{#1}}}}
\newcommand{\WarningTok}[1]{\textcolor[rgb]{0.56,0.35,0.01}{\textbf{\textit{#1}}}}
\newcommand{\AlertTok}[1]{\textcolor[rgb]{0.94,0.16,0.16}{#1}}
\newcommand{\ErrorTok}[1]{\textcolor[rgb]{0.64,0.00,0.00}{\textbf{#1}}}
\newcommand{\NormalTok}[1]{#1}
\usepackage{graphicx,grffile}
\makeatletter
\def\maxwidth{\ifdim\Gin@nat@width>\linewidth\linewidth\else\Gin@nat@width\fi}
\def\maxheight{\ifdim\Gin@nat@height>\textheight\textheight\else\Gin@nat@height\fi}
\makeatother
% Scale images if necessary, so that they will not overflow the page
% margins by default, and it is still possible to overwrite the defaults
% using explicit options in \includegraphics[width, height, ...]{}
\setkeys{Gin}{width=\maxwidth,height=\maxheight,keepaspectratio}
\IfFileExists{parskip.sty}{%
\usepackage{parskip}
}{% else
\setlength{\parindent}{0pt}
\setlength{\parskip}{6pt plus 2pt minus 1pt}
}
\setlength{\emergencystretch}{3em}  % prevent overfull lines
\providecommand{\tightlist}{%
  \setlength{\itemsep}{0pt}\setlength{\parskip}{0pt}}
\setcounter{secnumdepth}{0}
% Redefines (sub)paragraphs to behave more like sections
\ifx\paragraph\undefined\else
\let\oldparagraph\paragraph
\renewcommand{\paragraph}[1]{\oldparagraph{#1}\mbox{}}
\fi
\ifx\subparagraph\undefined\else
\let\oldsubparagraph\subparagraph
\renewcommand{\subparagraph}[1]{\oldsubparagraph{#1}\mbox{}}
\fi

%%% Use protect on footnotes to avoid problems with footnotes in titles
\let\rmarkdownfootnote\footnote%
\def\footnote{\protect\rmarkdownfootnote}

%%% Change title format to be more compact
\usepackage{titling}

% Create subtitle command for use in maketitle
\newcommand{\subtitle}[1]{
  \posttitle{
    \begin{center}\large#1\end{center}
    }
}

\setlength{\droptitle}{-2em}

  \title{615 Pre\_EDA}
    \pretitle{\vspace{\droptitle}\centering\huge}
  \posttitle{\par}
    \author{}
    \preauthor{}\postauthor{}
    \date{}
    \predate{}\postdate{}
  

\begin{document}
\maketitle

\subsection{Load Package \& Read the
Data}\label{load-package-read-the-data}

\begin{Shaded}
\begin{Highlighting}[]
\KeywordTok{library}\NormalTok{(tidyverse)}
\end{Highlighting}
\end{Shaded}

\begin{verbatim}
## -- Attaching packages --------------------------------------- tidyverse 1.2.1 --
\end{verbatim}

\begin{verbatim}
## √ ggplot2 3.0.0     √ purrr   0.2.5
## √ tibble  1.4.2     √ dplyr   0.7.6
## √ tidyr   0.8.1     √ stringr 1.3.1
## √ readr   1.1.1     √ forcats 0.3.0
\end{verbatim}

\begin{verbatim}
## -- Conflicts ------------------------------------------ tidyverse_conflicts() --
## x dplyr::filter() masks stats::filter()
## x dplyr::lag()    masks stats::lag()
\end{verbatim}

\begin{Shaded}
\begin{Highlighting}[]
\KeywordTok{library}\NormalTok{(scales) }
\end{Highlighting}
\end{Shaded}

\begin{verbatim}
## 
## Attaching package: 'scales'
\end{verbatim}

\begin{verbatim}
## The following object is masked from 'package:purrr':
## 
##     discard
\end{verbatim}

\begin{verbatim}
## The following object is masked from 'package:readr':
## 
##     col_factor
\end{verbatim}

\begin{Shaded}
\begin{Highlighting}[]
\KeywordTok{library}\NormalTok{(readr)}

\KeywordTok{library}\NormalTok{(dplyr)}

\KeywordTok{library}\NormalTok{(stats)}

\KeywordTok{library}\NormalTok{(factoextra) }\CommentTok{# clustering algorithms & visualization}
\end{Highlighting}
\end{Shaded}

\begin{verbatim}
## Welcome! Related Books: `Practical Guide To Cluster Analysis in R` at https://goo.gl/13EFCZ
\end{verbatim}

\begin{Shaded}
\begin{Highlighting}[]
\KeywordTok{library}\NormalTok{(corrplot)}
\end{Highlighting}
\end{Shaded}

\begin{verbatim}
## corrplot 0.84 loaded
\end{verbatim}

\begin{Shaded}
\begin{Highlighting}[]
\KeywordTok{library}\NormalTok{(graphics)}

\KeywordTok{library}\NormalTok{(cluster)    }\CommentTok{# clustering algorithms}

\NormalTok{frm_NCBI <-}\StringTok{ }\KeywordTok{read.csv}\NormalTok{(}\StringTok{'framingham.csv'}\NormalTok{) }\CommentTok{#kaggles}
\NormalTok{frm_ktrain <-}\StringTok{ }\KeywordTok{read_csv}\NormalTok{(}\StringTok{'frmgham2.csv'}\NormalTok{) }\CommentTok{#NIH}
\end{Highlighting}
\end{Shaded}

\begin{verbatim}
## Parsed with column specification:
## cols(
##   .default = col_integer(),
##   SYSBP = col_double(),
##   DIABP = col_double(),
##   BMI = col_double()
## )
\end{verbatim}

\begin{verbatim}
## See spec(...) for full column specifications.
\end{verbatim}

\begin{Shaded}
\begin{Highlighting}[]
\KeywordTok{problems}\NormalTok{(frm_NCBI)}
\end{Highlighting}
\end{Shaded}

\begin{verbatim}
## # tibble [0 x 4]
## # ... with 4 variables: row <int>, col <int>, expected <chr>, actual <chr>
\end{verbatim}

\begin{Shaded}
\begin{Highlighting}[]
\KeywordTok{problems}\NormalTok{(frm_ktrain)}
\end{Highlighting}
\end{Shaded}

\begin{verbatim}
## # tibble [0 x 4]
## # ... with 4 variables: row <int>, col <int>, expected <chr>, actual <chr>
\end{verbatim}

\subsection{Description of the
Datasets}\label{description-of-the-datasets}

The Framingham Heart Dataset The Framingham Heart Study is one of the
longest ongoing cardiovascular cohort study in the world. It started in
1948 and the initial number of adult subjects was 5,209. The study is
now its third generation and was conducted on the inhabitants of the
city of Framingham Massachusetts. The results from the study have been
used to better understand the epidemiology hypertensive or
arteriosclerotic cardiovascular disease. Prevailing knowledge/lifestyle
factors pertaining to heart disease such as diet, exercise and common
medications such as aspirin.

\begin{enumerate}
\def\labelenumi{\arabic{enumi}.}
\tightlist
\item
  The following demographic risk factors were included in the Framingham
  Heart Dataset:-
\item
  Sex: male or female:
\item
  Age: age of the patient:
\item
  Education: Different education Levels were coded 1 for some high
  school, 2 for a high school diploma or GED, 3 for some College or
  vocational school, and 4 for a college degree.
\item
  The data set also includes behavioral risk factors associated with
  smoking CurrentSmoker: If the patient is a current smoker or not
  CigsPerDay: The number of cigarettes that the person smoked on average
  in one day.
\item
  Medical history risk factors: BPMeds: Whether or not the patient was
  on blood pressure medication PrevalentStroke: Whether or not the
  patient had previously had a stroke PrevalentHyp: Whether or not the
  patient was hypertensive Diabetes: Whether or not the patient had
  diabetes
\item
  Risk factors from the first physical examination of the patient:
  TotChol: Total cholesterol levels SysBP: Systolic blood pressure
  DiaBP: Diastolic blood pressure BMI: Body Mass Index HeartRate: Heart
  rate Glucose: Glucose level TenYearCHD: 10 year risk of coronary heart
  disease CHD(This extra feature or Variable was only included in the
  Kaggle Dataset and not the NIH Dataset)
\end{enumerate}

\subsection{Principal Component Analysis
(PCA)}\label{principal-component-analysis-pca}

Description of the data exploration/analysis Principal Component
Analysis (PCA) A multivariate Analysis begins with having several
substantial correlated variables. In such an event, one can use a
statistical analysis called Principal Component Analysis (PCA). PCA
functions as a dimension-reduction tool that is often used to reduce a
large set of variables to a small set that still contains most of the
information in the large set. The smaller number of variables in the
small set is called principal components. The PCAs are numbered
1,2,3,4\ldots{}. The first PCA value generally accounts for as much of
the variability in the data as possible, and each succeeding component
accounts for as much of the remaining variability as possible.

We did a PCA on The Framingham Heart Datasets that were both obtained
from the NIH and Kaggles. Our reason for doing PCA was to find which
variables were the most important in the entire dataset: Which variable
contributed the most to disease risk all factors considered? After our
PCA analysis, we identified that Total Cholestrol levels were the most
important variable for this group of people. This can be explained by
the diet and lifestyle choices. Foods high in bad cholestrol overall put
you one at very high risk for cardiovascular diseases and heart health.

\begin{Shaded}
\begin{Highlighting}[]
\CommentTok{# Principal component analysis(PCA) #NIH}

\CommentTok{#frm_ktrain <- read.csv('framingham.csv')}
\NormalTok{frm_ktrain <-}\StringTok{ }\KeywordTok{read_csv}\NormalTok{(}\StringTok{'frmgham2.csv'}\NormalTok{)}
\end{Highlighting}
\end{Shaded}

\begin{verbatim}
## Parsed with column specification:
## cols(
##   .default = col_integer(),
##   SYSBP = col_double(),
##   DIABP = col_double(),
##   BMI = col_double()
## )
\end{verbatim}

\begin{verbatim}
## See spec(...) for full column specifications.
\end{verbatim}

\begin{Shaded}
\begin{Highlighting}[]
\NormalTok{res.pca <-}\StringTok{ }\KeywordTok{prcomp}\NormalTok{(}\KeywordTok{na.omit}\NormalTok{(frm_ktrain[,}\KeywordTok{c}\NormalTok{(}\DecValTok{2}\OperatorTok{:}\DecValTok{6}\NormalTok{, }\DecValTok{8}\OperatorTok{:}\DecValTok{14}\NormalTok{)], }\DataTypeTok{center =}\NormalTok{ TRU, }\DataTypeTok{scale =} \OtherTok{TRUE}\NormalTok{)) }\CommentTok{#plot1}

\KeywordTok{fviz_eig}\NormalTok{(res.pca) }\CommentTok{#plot one}
\end{Highlighting}
\end{Shaded}

\includegraphics{PCAforNIH_files/figure-latex/unnamed-chunk-2-1.pdf}

\begin{Shaded}
\begin{Highlighting}[]
\KeywordTok{fviz_pca_biplot}\NormalTok{(res.pca) }\CommentTok{#plot two}
\end{Highlighting}
\end{Shaded}

\includegraphics{PCAforNIH_files/figure-latex/unnamed-chunk-2-2.pdf}

\begin{Shaded}
\begin{Highlighting}[]
\KeywordTok{fviz_pca_ind}\NormalTok{(res.pca) }\CommentTok{#gives a scatter plot. very messy}
\end{Highlighting}
\end{Shaded}

\includegraphics{PCAforNIH_files/figure-latex/unnamed-chunk-2-3.pdf}

\begin{Shaded}
\begin{Highlighting}[]
\KeywordTok{summary}\NormalTok{(res.pca) ##As you can see, principal components 1 and 2 have the highest standard deviation, use them}
\end{Highlighting}
\end{Shaded}

\begin{verbatim}
## Importance of components:
##                            PC1     PC2     PC3      PC4     PC5     PC6
## Standard deviation     45.4627 26.5103 22.7142 12.94025 11.4519 9.23998
## Proportion of Variance  0.5548  0.1887  0.1385  0.04495  0.0352 0.02292
## Cumulative Proportion   0.5548  0.7435  0.8820  0.92692  0.9621 0.98505
##                            PC7     PC8     PC9    PC10    PC11    PC12
## Standard deviation     6.35481 3.73987 1.01090 0.46518 0.25779 0.17051
## Proportion of Variance 0.01084 0.00375 0.00027 0.00006 0.00002 0.00001
## Cumulative Proportion  0.99589 0.99964 0.99992 0.99997 0.99999 1.00000
\end{verbatim}

\subsection{PCA Discussion}\label{pca-discussion}

From our analysis of The Framingham Heart Dataset(NIH): The Histogram
plot obtained from {[}fviz\_eig(res.pca){]} which plots the highest
eigenvectors which are the highest PCA shows that PCA1 explains over
60\% of the variance in the dataset and subsequently PCA2 explains over
\%19 and PCA3 explaining the remaining PCAs explaining the remaining
variance.

Summary Importance of PCAs:The summary shows what we expected from the
PCA plot1. PCA which is what we picked to explain the most important
variables in the Dataset had the highest standard deviation, proportion
of variance and cumulative proportion followed PCA2 which had half of
these values.

mportance of components: PC1 PC2 PC3 PC4 PC5 PC6 PC7 Standard deviation
45.4627 26.5103 22.7142 12.94025 11.4519 9.23998 6.35481 Proportion of
Variance 0.5548 0.1887 0.1385 0.04495 0.0352 0.02292 0.01084 Cumulative
Proportion 0.5548 0.7435 0.8820 0.92692 0.9621 0.98505 0.99589 PC8 PC9
PC10 PC11 PC12 Standard deviation 3.73987 1.01090 0.46518 0.25779
0.17051 Proportion of Variance 0.00375 0.00027 0.00006 0.00002 0.00001
Cumulative Proportion 0.99964 0.99992 0.99997 0.99999 1.00000

Plot2: This plot shows: For PCA1: TotalChol is the most important
variable to PCA1(Dim2).

For PCA2:Glucose, SysBp, DiasBp, HeartRate, BMI and CIGPDAY were the
most important variables in PCA2(Dim2)

Our reason for doing PCA was to find which variables were the most
important in the entire dataset: Which variable contributed the most to
disease risk all factors considered? After our PCA analysis, we
identified that Total Cholestrol levels were the most important variable
for this group of people. This can be explained by the diet and
lifestyle choices. Foods high in bad cholestrol overall put you one at
very high risk for cardiovascular diseases and heart health.

\begin{Shaded}
\begin{Highlighting}[]
\CommentTok{#VARIATION PLOT for VARIANCE #NIH DATA}
\CommentTok{#frm_ktrain <- read_csv('frmgham2.csv') #used this}
\CommentTok{#frm_NCBI <- read.csv('framingham.csv') }
\NormalTok{var <-}\StringTok{ }\KeywordTok{get_pca_var}\NormalTok{(res.pca)}
\KeywordTok{corrplot}\NormalTok{(var}\OperatorTok{$}\NormalTok{cos2, }\DataTypeTok{is.corr=}\OtherTok{FALSE}\NormalTok{)}
\end{Highlighting}
\end{Shaded}

\includegraphics{PCAforNIH_files/figure-latex/unnamed-chunk-3-1.pdf}

\begin{Shaded}
\begin{Highlighting}[]
\NormalTok{frm_ktrain.clean <-frm_ktrain }\OperatorTok\StringTok{ }\KeywordTok{mutate_at}\NormalTok{(}\KeywordTok{vars}\NormalTok{(SEX,CURSMOKE, DIABETES, BPMEDS, educ, PREVCHD, PREVAP, PREVMI, PREVSTRK, PREVHYP, PERIOD, DEATH, }
\NormalTok{                                                  ANGINA, HOSPMI, MI_FCHD, ANYCHD, STROKE, CVD, HYPERTEN) , }\KeywordTok{funs}\NormalTok{(}\KeywordTok{factor}\NormalTok{(.)))}
\KeywordTok{summary}\NormalTok{(frm_ktrain.clean)}
\end{Highlighting}
\end{Shaded}

\begin{verbatim}
##      RANDID        SEX         TOTCHOL           AGE       
##  Min.   :   2448   1:5022   Min.   :107.0   Min.   :32.00  
##  1st Qu.:2474378   2:6605   1st Qu.:210.0   1st Qu.:48.00  
##  Median :5006008            Median :238.0   Median :54.00  
##  Mean   :5004741            Mean   :241.2   Mean   :54.79  
##  3rd Qu.:7472730            3rd Qu.:268.0   3rd Qu.:62.00  
##  Max.   :9999312            Max.   :696.0   Max.   :81.00  
##                             NA's   :409                    
##      SYSBP           DIABP        CURSMOKE    CIGPDAY           BMI       
##  Min.   : 83.5   Min.   : 30.00   0:6598   Min.   : 0.00   Min.   :14.43  
##  1st Qu.:120.0   1st Qu.: 75.00   1:5029   1st Qu.: 0.00   1st Qu.:23.09  
##  Median :132.0   Median : 82.00            Median : 0.00   Median :25.48  
##  Mean   :136.3   Mean   : 83.04            Mean   : 8.25   Mean   :25.88  
##  3rd Qu.:149.0   3rd Qu.: 90.00            3rd Qu.:20.00   3rd Qu.:28.07  
##  Max.   :295.0   Max.   :150.00            Max.   :90.00   Max.   :56.80  
##                                            NA's   :79      NA's   :52     
##  DIABETES   BPMEDS         HEARTRTE         GLUCOSE         educ     
##  0:11097   0   :10090   Min.   : 37.00   Min.   : 39.00   1   :4690  
##  1:  530   1   :  944   1st Qu.: 69.00   1st Qu.: 72.00   2   :3410  
##            NA's:  593   Median : 75.00   Median : 80.00   3   :1885  
##                         Mean   : 76.78   Mean   : 84.12   4   :1347  
##                         3rd Qu.: 85.00   3rd Qu.: 89.00   NA's: 295  
##                         Max.   :220.00   Max.   :478.00              
##                         NA's   :6        NA's   :1440                
##  PREVCHD   PREVAP    PREVMI    PREVSTRK  PREVHYP       TIME      PERIOD  
##  0:10785   0:11000   0:11253   0:11475   0:6283   Min.   :   0   1:4434  
##  1:  842   1:  627   1:  374   1:  152   1:5344   1st Qu.:   0   2:3930  
##                                                   Median :2156   3:3263  
##                                                   Mean   :1957           
##                                                   3rd Qu.:4252           
##                                                   Max.   :4854           
##                                                                          
##       HDLC             LDLC       DEATH    ANGINA   HOSPMI    MI_FCHD 
##  Min.   : 10.00   Min.   : 20.0   0:8100   0:9725   0:10473   0:9839  
##  1st Qu.: 39.00   1st Qu.:145.0   1:3527   1:1902   1: 1154   1:1788  
##  Median : 48.00   Median :173.0                                       
##  Mean   : 49.37   Mean   :176.5                                       
##  3rd Qu.: 58.00   3rd Qu.:205.0                                       
##  Max.   :189.00   Max.   :565.0                                       
##  NA's   :8600     NA's   :8601                                        
##  ANYCHD   STROKE    CVD      HYPERTEN     TIMEAP         TIMEMI    
##  0:8469   0:10566   0:8728   0:2985   Min.   :   0   Min.   :   0  
##  1:3158   1: 1061   1:2899   1:8642   1st Qu.:6224   1st Qu.:7212  
##                                       Median :8766   Median :8766  
##                                       Mean   :7242   Mean   :7594  
##                                       3rd Qu.:8766   3rd Qu.:8766  
##                                       Max.   :8766   Max.   :8766  
##                                                                    
##     TIMEMIFC       TIMECHD        TIMESTRK       TIMECVD    
##  Min.   :   0   Min.   :   0   Min.   :   0   Min.   :   0  
##  1st Qu.:7050   1st Qu.:5598   1st Qu.:7295   1st Qu.:6004  
##  Median :8766   Median :8766   Median :8766   Median :8766  
##  Mean   :7543   Mean   :7008   Mean   :7661   Mean   :7166  
##  3rd Qu.:8766   3rd Qu.:8766   3rd Qu.:8766   3rd Qu.:8766  
##  Max.   :8766   Max.   :8766   Max.   :8766   Max.   :8766  
##                                                             
##     TIMEDTH        TIMEHYP    
##  Min.   :  26   Min.   :   0  
##  1st Qu.:7798   1st Qu.:   0  
##  Median :8766   Median :2429  
##  Mean   :7854   Mean   :3599  
##  3rd Qu.:8766   3rd Qu.:7329  
##  Max.   :8766   Max.   :8766  
## 
\end{verbatim}

\subsection{Discussion}\label{discussion}

We plotted a different plot so as to better visualize the significance
of these variables after a PCA analysis. The plot confirms our previous
analysis and results where Total Cholestrol levels were identified to be
the most important variable for this group of people in regards to PCA1.
SysBP and Glucose captured the remaining variance in PCA2 while
CIGPERDAY was significant in PCA3 and last of all HeartRate was
significant in PCA4. As a rule of thumb, we go with the top two or three
variables which capture most of the variance in the dataset. The reason
for Total Cholestrol levels being high could be explained by dietary
factors for this particular group: Foods high in bad cholestrol overall
put you one at very high risk for cardiovascular diseases and heart
health. It could be that the consume a diet high in cholestrol and do
little to lower it.

\begin{Shaded}
\begin{Highlighting}[]
\CommentTok{#my favorite one}
\CommentTok{#Panel correlation for NIH Dataset(frm_ktrain)}
\CommentTok{#gives correlation between variables}
\NormalTok{panel.cor <-}\StringTok{ }\ControlFlowTok{function}\NormalTok{(x, y, }\DataTypeTok{digits =} \DecValTok{2}\NormalTok{, }\DataTypeTok{prefix =} \StringTok{""}\NormalTok{, cex.cor, ...)}

\NormalTok{\{}

\NormalTok{  usr <-}\StringTok{ }\KeywordTok{par}\NormalTok{(}\StringTok{"usr"}\NormalTok{); }\KeywordTok{on.exit}\NormalTok{(}\KeywordTok{par}\NormalTok{(usr))}

  \KeywordTok{par}\NormalTok{(}\DataTypeTok{usr =} \KeywordTok{c}\NormalTok{(}\DecValTok{0}\NormalTok{, }\DecValTok{1}\NormalTok{, }\DecValTok{0}\NormalTok{, }\DecValTok{1}\NormalTok{))}

\NormalTok{  r <-}\StringTok{ }\KeywordTok{abs}\NormalTok{(}\KeywordTok{cor}\NormalTok{(x, y))}

\NormalTok{  txt <-}\StringTok{ }\KeywordTok{format}\NormalTok{(}\KeywordTok{c}\NormalTok{(r, }\FloatTok{0.123456789}\NormalTok{), }\DataTypeTok{digits =}\NormalTok{ digits)[}\DecValTok{1}\NormalTok{]}

\NormalTok{  txt <-}\StringTok{ }\KeywordTok{paste0}\NormalTok{(prefix, txt)}

  \ControlFlowTok{if}\NormalTok{(}\KeywordTok{missing}\NormalTok{(cex.cor)) cex.cor <-}\StringTok{ }\FloatTok{0.8}\OperatorTok{/}\KeywordTok{strwidth}\NormalTok{(txt)}

  \KeywordTok{text}\NormalTok{(}\FloatTok{0.5}\NormalTok{, }\FloatTok{0.5}\NormalTok{, txt, }\DataTypeTok{cex =}\NormalTok{ cex.cor }\OperatorTok{*}\StringTok{ }\NormalTok{r)}

\NormalTok{\}}


\CommentTok{#frm_ktrain <- read_csv('frmgham2.csv')}

\KeywordTok{pairs}\NormalTok{( }\OperatorTok{~}\NormalTok{TOTCHOL}\OperatorTok{+}\NormalTok{AGE}\OperatorTok{+}\NormalTok{SYSBP}\OperatorTok{+}\NormalTok{DIABP}\OperatorTok{+}\NormalTok{BMI}\OperatorTok{+}\NormalTok{HEARTRTE }\OperatorTok{+}\NormalTok{GLUCOSE}\OperatorTok{+}\NormalTok{CIGPDAY, }\DataTypeTok{data=}\NormalTok{frm_ktrain, }\DataTypeTok{na.action=}\NormalTok{na.omit,}

       \DataTypeTok{lower.panel =}\NormalTok{ panel.smooth,}

       \DataTypeTok{upper.panel =}\NormalTok{ panel.cor)}
\end{Highlighting}
\end{Shaded}

\includegraphics{PCAforNIH_files/figure-latex/unnamed-chunk-4-1.pdf}
\#\# Correlation Plot Reason for Analysis: We generated a correlation
plot. The purpose of the correlation plot was to show how much one
variable is affected by another. The relationship between two variables
is called their correlation.

Results: From the results of our analysis, it was clear that SYSBP and
DIASBP are the most highly positively correlated variables in the entire
dataset.That means if one increased then the other would increase and
vice-versa. The correlation value was 0.72. However, SYSBP is a better
variable for predicting cardiovacular health than DIABP. The next
slightly correlated variables were Age and SysBP which had a low
correlation value of 0.38. The DIABP and BMI had an almost similar value
of 0.34. This confirms the assumption that as you grow older, you run a
slightly higher risk of increasing your SYSBP.

Discussion: Bloop Pressure: A patients Blood Pressure reading is taken
in two readings. A brief description from the CDC webiste states that
the first number, called systolic blood pressure, measures the pressure
in your blood vessels when your heart beats. The second number, called
diastolic blood pressure, measures the pressure in your blood vessels
when your heart rests between beats. The general assumption is that as
one increases the other would consequentially increase. The correlation
plots confirmed this assumption.

The next slightly correlated variables were Age and SysBP which had a
low correlation value of 0.38. The DIABP and BMI had an almost similar
value of 0.34. This confirms the assumption that as you grow older, you
run a slightly higher risk of increasing your SYSBP. It makes sense
because older population have higher cases of High Blood pressure as
opposed to younger populations.

\subsection{PCA FOR KAGGLES DATASET WITH LESS
VARIABLES}\label{pca-for-kaggles-dataset-with-less-variables}

\begin{Shaded}
\begin{Highlighting}[]
\CommentTok{# Principal component analysis #Kaggles}

\NormalTok{frm_NCBI <-}\StringTok{ }\KeywordTok{read.csv}\NormalTok{(}\StringTok{'framingham.csv'}\NormalTok{) }\CommentTok{#kaggles}

\NormalTok{res.pca <-}\StringTok{ }\KeywordTok{prcomp}\NormalTok{(}\KeywordTok{na.omit}\NormalTok{(frm_NCBI[,}\KeywordTok{c}\NormalTok{(}\DecValTok{2}\OperatorTok{:}\DecValTok{6}\NormalTok{, }\DecValTok{8}\OperatorTok{:}\DecValTok{14}\NormalTok{)], }\DataTypeTok{center =}\NormalTok{ TRU, }\DataTypeTok{scale =} \OtherTok{TRUE}\NormalTok{))}


\KeywordTok{fviz_eig}\NormalTok{(res.pca) }\CommentTok{#plot one}
\end{Highlighting}
\end{Shaded}

\includegraphics{PCAforNIH_files/figure-latex/unnamed-chunk-5-1.pdf}

\begin{Shaded}
\begin{Highlighting}[]
\KeywordTok{fviz_pca_biplot}\NormalTok{(res.pca) }\CommentTok{#plot two}
\end{Highlighting}
\end{Shaded}

\includegraphics{PCAforNIH_files/figure-latex/unnamed-chunk-5-2.pdf}

\begin{Shaded}
\begin{Highlighting}[]
\KeywordTok{fviz_pca_ind}\NormalTok{(res.pca) }\CommentTok{#gives a scatter plot. very messy}
\end{Highlighting}
\end{Shaded}

\includegraphics{PCAforNIH_files/figure-latex/unnamed-chunk-5-3.pdf}

\begin{Shaded}
\begin{Highlighting}[]
\KeywordTok{summary}\NormalTok{(res.pca) ##As you can see, principal components 1 and 2 have the highest standard deviation, use them}
\end{Highlighting}
\end{Shaded}

\begin{verbatim}
## Importance of components:
##                            PC1     PC2      PC3      PC4     PC5     PC6
## Standard deviation     44.5331 23.5903 12.51630 11.27559 7.90686 6.21363
## Proportion of Variance  0.6746  0.1893  0.05329  0.04325 0.02127 0.01313
## Cumulative Proportion   0.6746  0.8639  0.91723  0.96048 0.98175 0.99488
##                            PC7     PC8     PC9    PC10    PC11    PC12
## Standard deviation     3.71413 0.99459 0.32717 0.31287 0.16153 0.15499
## Proportion of Variance 0.00469 0.00034 0.00004 0.00003 0.00001 0.00001
## Cumulative Proportion  0.99958 0.99991 0.99995 0.99998 0.99999 1.00000
\end{verbatim}

\section{VARIATION PLOT for VARIANCE \#Kaggles
DATA}\label{variation-plot-for-variance-kaggles-data}

\section{\texorpdfstring{frm\_NCBI \textless{}-
read.csv(`framingham.csv') \#Kaggles change to
this}{frm\_NCBI \textless{}- read.csv(framingham.csv) \#Kaggles change to this}}\label{frm_ncbi---read.csvframingham.csv-kaggles-change-to-this}

\begin{Shaded}
\begin{Highlighting}[]
\CommentTok{#VARIATION PLOT for VARIANCE #Kaggles DATA}
\CommentTok{#frm_NCBI <- read.csv('framingham.csv') #Kaggles change to this}

\NormalTok{var <-}\StringTok{ }\KeywordTok{get_pca_var}\NormalTok{(res.pca)}
\KeywordTok{corrplot}\NormalTok{(var}\OperatorTok{$}\NormalTok{cos2, }\DataTypeTok{is.corr=}\OtherTok{FALSE}\NormalTok{)}
\end{Highlighting}
\end{Shaded}

\includegraphics{PCAforNIH_files/figure-latex/unnamed-chunk-6-1.pdf}

\begin{Shaded}
\begin{Highlighting}[]
\NormalTok{frm_NCBI.clean <-frm_NCBI }\OperatorTok\StringTok{ }\KeywordTok{mutate_at}\NormalTok{(}\KeywordTok{vars}\NormalTok{(male,age,education,currentSmoker,cigsPerDay,BPMeds,prevalentStroke,prevalentHyp,diabetes,totChol,sysBP,diaBP,BMI,heartRate,glucose,TenYearCHD) , }\KeywordTok{funs}\NormalTok{(}\KeywordTok{factor}\NormalTok{(.)))}
\KeywordTok{summary}\NormalTok{(frm_NCBI.clean)}
\end{Highlighting}
\end{Shaded}

\begin{verbatim}
##  male          age       education   currentSmoker   cigsPerDay  
##  0:2420   40     : 192   1   :1720   0:2145        0      :2145  
##  1:1820   46     : 182   2   :1253   1:2095        20     : 734  
##           42     : 180   3   : 689                 30     : 218  
##           41     : 174   4   : 473                 15     : 210  
##           48     : 173   NA's: 105                 10     : 143  
##           39     : 170                             (Other): 761  
##           (Other):3169                             NA's   :  29  
##   BPMeds     prevalentStroke prevalentHyp diabetes    totChol    
##  0   :4063   0:4215          0:2923       0:4131   240    :  85  
##  1   : 124   1:  25          1:1317       1: 109   220    :  70  
##  NA's:  53                                         260    :  62  
##                                                    210    :  61  
##                                                    232    :  59  
##                                                    (Other):3853  
##                                                    NA's   :  50  
##      sysBP          diaBP           BMI         heartRate   
##  120    : 107   80     : 262   22.19  :  18   75     : 563  
##  130    : 102   82     : 152   22.54  :  18   80     : 385  
##  110    :  96   85     : 137   22.91  :  18   70     : 305  
##  115    :  89   70     : 135   23.48  :  18   60     : 231  
##  125    :  88   81     : 131   23.09  :  16   85     : 228  
##  124    :  84   84     : 122   (Other):4133   (Other):2527  
##  (Other):3674   (Other):3301   NA's   :  19   NA's   :   1  
##     glucose     TenYearCHD
##  75     : 193   0:3596    
##  77     : 167   1: 644    
##  73     : 156             
##  80     : 153             
##  70     : 152             
##  (Other):3031             
##  NA's   : 388
\end{verbatim}

\subsection{Discussion}\label{discussion-1}

It was interesting to note that for the kaggles dataset that had less
variables overall and one new variable that predicts the person risks
for TenYearCHD.Total Cholestrol overall was the most important variable
in PCA1. SYSBP the most important variable in PCA2 that captured the
remaining variance that PCA1 did not capture: PCA1 accounts for most of
the variance. It was also interesting to note that upon adding GLUCOSE
as a variable like in the entire dataset with all the variables
considered, glucose was an important predictor for cardiovascular
diseases. This confrims our assumptions and prevailing knowledge that
chronic high glucose levels putting you at a higher risk for DIABETES
AND High Blood Pressure diseases.


\end{document}
